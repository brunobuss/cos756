\documentclass[11pt,a4paper]{article}
% mrc April 30,2011

%%Pacotes utilizados para a localização do texto em pt-br
\usepackage[portuguese,brazil]{babel}
\usepackage[utf8]{inputenc}
\usepackage[T1]{fontenc}

%%Pacotes relacionados a escrita matemática e algoritmos
\usepackage{amsmath}
\usepackage{amsfonts}
\usepackage{amsthm}
\usepackage{amssymb}

%%Pacotes relacionados a formatação do texto
\usepackage{indentfirst}

%%Pacotes relacionados a imagens
\usepackage{graphicx}


\title{Tracking de uma bola em um jogo de futebol}
\author{Bruno Buss e Lucas Pierezan}

\begin{document}
\maketitle

\begin{abstract}
 Este relatório diz respeito ao trabalho de realizar o \textit{tracking} de uma bola em jogo de futebol. Apresentamos aqui as ideias e métodos utilizados bem como dificuldades encontradas.
\end{abstract}

\section{Introdução}
 Nesse trabalho foi implementado um algoritmo para realizar o \textit{tracking} de uma bola em um jogo de futebol. Este problema tem diversas aplicações como a detecção automática de eventos do jogo (como gol e impedimento) e geração de dados estatísticos de interesse de emissoras de tv e times de futebol. Na literatura podemos encotrar alguns trabalhos sobre esse assunto \cite{}\cite{}\cite{} indicando um aspécto desafiador devido a fênomenos como oclusão e distorção da bola em alta velocidade. Baseamos nosso trabalho principalmente em ideias presentes em \cite{}. Outras abordagens como \cite{} e \cite{} utilizam como sub-procedimento a detecção de jogadores o que aumentaria consideravelmente a complexidade do trabalho.

Como em \cite{} utilizamos a Tranformada Circular de Hough (CHT) como ferramenta principal. A CHT é um método clássico utilizado para achar padrões circulares em imagens \cite{}. Existem muitas adaptações da tranformada\cite{} e buscamos incorporar um conjunto de adaptações priorizando os aspéctos particulares do nosso problema. Utilizamos a CHT como procedimento para gerar um valor de quão circular é uma região, para posterior filtragem. Na seção 2 explicamos com mais detalhe como isso é feito.

Além da CHT, utilizamos também uma análise de similaridade de histograma para gerar um valor de quão similar é o histograma de uma região comparado com o histograma da bola, que supomos saber. Com os valores de circularidade e similaridade de histograma geramos um \textit{score} final que é então utilizado para localizar a bola. Na seção 3 explicamos como essa análise de histograma é realizada e como é gerado o \textit{score} final.

Na seção 3 explicamos como utilizamos o conceito de Região de Interesse (ROI) para restringir a busca da bola usando informações obtidas em iterações passadas do algoritmo. O comportamento dessa ROI é descrito em detalhes bem como os problemas gerador por essa abordagem.

Vale ressaltar que o algoritmo foi implementado em C++ e utilizamos a biblioteca OpenCV para efetuar alguns sub-procedimentos de processamento de imagens que julgamos secundários.

\section{Calculando valor de circularidade}

 Como já foi dito, para gerar o valor de circularidade, utilizamos a Transformada Circular de Hough (CHT). A CHT trabalha no espaço das arestas, ou seja, buscando padrões circulares no espaço das arestas. Aqui já podemos ver a hipotese da bola estar relativamente bem contrastada na imagem, aparecendo então no espaço das arestas.

 Para achar as arestas da imagem $I$, em RGB, primeiramente convertemos essa para a imagem $G_I$ em escalas de cinza através da função ``cvtColor`` do openCV. Aplicamos também um borramento gaussiano em $G_I$ para eliminar possíveis ruidos usando a função ''GaussianBlur``. Finalmente utilizamos a função ''Canny``. 

 A CHT busca círcunferências no espaço das arestas por um esquema de votação em um espaço acumulador. O espaço acumulador é o espaço dos parâmetros da circunferência que estamos buscando e é da forma $a completar$. Cada ponto no espaço acumulador representa uma possível circunferência e portanto é da forma $(cx,cy,R)$. Cada ponto de aresta $P = (x,y)$ e raio $R$ vota, no espaço acumulador, nos pontos de centro que poderiam que gerar o ponto $P$ como parte de uma circunferência. De fato, na implementação clássica da CHT, para cada ponto de aresta $P$ e raio $R$ é realizada uma votação no conjunto de pontos $S_{P,R} = ( (cx,cy,R) \mid (cx - x)^2 + (cy - y)^2 = R^2)$ no espaço acumulador. O método consiste então em processar todos os pontos de arestas, para os valores $R$ de raio que estão sendo considerados, e acumular votos no espaço dos parâmetros. Os pontos, no espaço dos parâmetros, com maior número de votos são apontados como centros de circunferêmcias. 

 Em nossa implementação, o espaço acumulador é representado pela \textit{struct} \textit{acumulator}. Essa \textit{struct} é basicamente uma matriz tridemensional $m$ com as informações do $minR$ e $maxR$. Essa matriz $m$ contém inteiros, que indicam o quantidade de votos recebida por um ponto. 

\section{Calculando valor de similaridade de histograma}
\section{Comportamento da Região de Interesse}
\section{Considerações Finais}


\begin{thebibliography}{99}



\end{thebibliography}



\end{document}